% !TEX program = xelatex
\documentclass[twocolumn]{article}
\usepackage{ctex}
\usepackage{amsmath}
\usepackage{graphicx}
\usepackage{xcolor}
\usepackage{listings}
\usepackage{geometry}
\usepackage{fancyhdr}
\usepackage{tabularx}
\usepackage{booktabs}
\usepackage{hyperref}

% --- 页面布局 ---
\geometry{a4paper, left=2cm, right=2cm, top=2.5cm, bottom=2.5cm, columnsep=1cm}

% --- 代码高亮配置 ---
\definecolor{codegreen}{rgb}{0,0.6,0}
\definecolor{codegray}{rgb}{0.5,0.5,0.5}
\definecolor{codepurple}{rgb}{0.58,0,0.82}
\definecolor{backcolour}{rgb}{0.95,0.95,0.92}

\lstdefinestyle{mystyle}{
    backgroundcolor=\color{backcolour},   
    commentstyle=\color{codegreen},
    keywordstyle=\color{magenta},
    numberstyle=\tiny\color{codegray},
    stringstyle=\color{codepurple},
    basicstyle=\ttfamily\footnotesize,
    breakatwhitespace=false,         
    breaklines=true,                 
    captionpos=b,                    
    keepspaces=true,                 
    numbers=left,                    
    numbersep=5pt,                  
    showspaces=false,                
    showstringspaces=false,
    showtabs=false,                  
    tabsize=2,
    columns=flexible,
    literate={\_}{{\textunderscore}}1,
    mathescape=false,
    texcl=false
}
\lstset{style=mystyle}

% --- 自定义 listings 语言 ---
\lstdefinelanguage{flex}{
    keywords={%
        options, case-insensitive, noyywrap, yylineno,
        x, s,
    },
    sensitive=false,
    morecomment=[l][\color{codegreen}]{//},
    morecomment=[s][\color{codegreen}]{/*}{*/},
    morecomment=[s][\color{codegreen}]{\%\{}{\%\}},
    morestring=[b][\color{codepurple}]",
    alsoletter={-,>},
    morekeywords=[2]{BEGIN, ECHO, REJECT, yylval, yytext, yyleng, yylex, curr_lineno},
    mathescape=false,
    texcl=false,
}

\lstdefinelanguage{cool}{
    keywords={%
        class, inherits, if, then, else, fi, while, loop, pool, let, in, case, of, esac, new, isvoid, not, true, false,
        Int, String, Bool, Object, SELF_TYPE, self
    },
    sensitive=false,
    morecomment=[l][\color{codegreen}]{--},
    morecomment=[s][\color{codegreen}]{(*}{*)},
    morestring=[b][\color{codepurple}]",
    morekeywords=[2]{},
}


% --- 页眉页脚 ---
\pagestyle{fancy}
\fancyhf{}
\fancyhead[L]{COOL 词法分析器实验报告}
\fancyhead[R]{\thepage}
\fancyfoot[C]{\small \copyright\ 2025}
\renewcommand{\headrulewidth}{0.4pt}
\renewcommand{\footrulewidth}{0.4pt}

% --- 标题 ---
\title{
    \vspace{-1cm} % 调整标题位置
    \textbf{COOL 语言词法分析器开发报告} \\
    \large \texttt{Compiler Principle Assignment}
}
\author{
    姓名: \textcolor{red}{在此处填写你的姓名} \\
    学号: \textcolor{red}{在此处填写你的学号} \\
    班级: \textcolor{red}{在此处填写你的班级}
}
\date{\today}

% --- 文档开始 ---
\begin{document}

\maketitle
\thispagestyle{fancy} % 首页也使用fancy样式

\begin{abstract}
\noindent
(自己根据情况改改)
本文档详细记录了 COOL (Classroom Object-Oriented Language) 词法分析器的设计与实现过程。报告首先深入阐述Flex词法分析器的工作原理,包括有限状态自动机理论、模式匹配机制和状态转换过程;然后详细说明词法规则的实现方法,涵盖关键字、标识符、字符串、注释及错误处理;最后通过完整的测试验证,包括集成测试,展示词法分析器与编译器其他组件的协作以及最终程序的正确运行。

\textcolor{red}{%
% TODO: 请根据你的实际工作,简要修改摘要内容。特别注意强调你对Flex原理的理解。
}
\end{abstract}

\section{评分细则与报告要求}
\label{sec:grading}
本报告的最终得分将基于以下几个方面。请确保您的报告和代码完整覆盖了所有评分项。

\textbf{特别说明:}
\begin{itemize}
    \item \textbf{Flex原理阐述}(20分)和\textbf{集成测试}(15分)是本次作业的重点,请务必认真完成
    \item 代码查重将使用专业查重工具检测,查重率达到100\%及以上将导致整个报告得0分
    \item 鼓励独立思考和原创性工作,可以参考资料但必须用自己的话表述
    \item 教师主观评分(10分)将根据报告整体质量、理解深度、创新性等综合评定
\end{itemize}

\begin{table*}[ht]
    \centering
    \caption{COOL 词法分析器评分标准}
    \label{tab:grading}
    \begin{tabularx}{\textwidth}{l X c}
        \toprule
        \textbf{类别} & \textbf{评分项说明} & \textbf{分数} \\
        \midrule
        \textbf{理论理解} & & \\
        \cmidrule(r){1-1}
        Flex工作原理 & 详细阐述Flex的工作流程、有限状态自动机(DFA/NFA)原理、模式匹配机制、状态转换过程。要求有清晰的图示或流程说明。 & 20 \\
        \midrule
        \textbf{基础功能实现} & & \\
        \cmidrule(r){1-1}
        关键字识别 & 正确识别所有COOL关键字,处理大小写不敏感。布尔常量首字母必须小写。 & 8 \\
        标识符与常量 & 正确区分 TYPE\_ID 和 OBJECT\_ID,正确解析整数和布尔常量。 & 8 \\
        操作符与注释 & 完整处理多字符操作符、单行注释、嵌套多行注释。 & 7 \\
        \midrule
        \textbf{核心难点} & & \\
        \cmidrule(r){1-1}
        字符串处理 & 使用状态机正确处理字符串,支持转义字符,处理各种字符串错误。 & 10 \\
        错误处理 & 检测并报告所有类型的词法错误(未闭合字符串/注释、字符串过长、非法字符等)。 & 7 \\
        \midrule
        \textbf{测试验证} & & \\
        \cmidrule(r){1-1}
        集成测试 & 完整的编译流程展示(词法→语法→语义→代码生成→SPIM运行),测试用例设计合理,输出结果完整正确。 & 15 \\
        \midrule
        \textbf{报告与代码} & & \\
        \cmidrule(r){1-1}
        报告质量 & 报告结构清晰,实现说明详细,测试结果完整。 & 5 \\
        代码质量 & cool.flex代码风格良好,逻辑清晰,注释适当。 & 5 \\
        \midrule
        \textbf{其他} & & \\
        \cmidrule(r){1-1}
        教师主观评分 & 整体完成度、创新性、深度理解等综合评价。 & 10 \\
        代码查重 & 查重率<50\%得5分;60\%得4分;70\%得3分;80\%得2分;90\%得1分;100\%及以上整个报告0分。 & 5 \\
        \bottomrule
        \multicolumn{2}{r}{\textbf{总分}} & \textbf{100} \\
    \end{tabularx}
\end{table*}

\section{项目概述与环境}
\subsection{项目目标}
(自己根据情况改改)
本次作业的目标是使用词法分析器生成工具 Flex 为 COOL 语言设计并实现一个完整的词法分析器。该分析器需要能够将 COOL 源代码文本文件转换为一个词法单元(Token)序列,并能正确处理各种边界情况和词法错误。

\subsection{开发环境}

\textcolor{red}{%
% TODO: 请详细填写以下开发环境信息
}

\subsubsection{硬件配置}
\begin{itemize}
    \item \textbf{CPU}: \textcolor{red}{例如:Intel Core i7-10700K @ 3.80GHz}
    \item \textbf{内存}: \textcolor{red}{例如:16GB DDR4}
    \item \textbf{硬盘}: \textcolor{red}{例如:512GB SSD}
\end{itemize}

\subsubsection{软件环境}
\begin{itemize}
    \item \textbf{操作系统}: \textcolor{red}{例如:Ubuntu 22.04.3 LTS / Windows 11 Pro 22H2 / macOS Ventura 13.4}
    \item \textbf{内核版本}: \textcolor{red}{运行 uname -r 或 ver 查看}
    \item \textbf{Flex 版本}: \textcolor{red}{运行 flex --version,例如:flex 2.6.4}
    \item \textbf{G++ 版本}: \textcolor{red}{运行 g++ --version,例如:g++ (Ubuntu 11.4.0-1ubuntu1) 11.4.0}
    \item \textbf{Make 版本}: \textcolor{red}{运行 make --version}
    \item \textbf{其他工具}: \textcolor{red}{SPIM版本、编辑器等}
\end{itemize}

\subsubsection{项目目录结构}
\textcolor{red}{%
% TODO: 使用 tree 命令或手动列出你的项目目录结构
}
\begin{verbatim}
/usr/class/assignments/PA2/(根据你情况)
|-- cool.flex       
|-- test.cl           
|-- lexer             
|-- cool-lex.cc       
|-- Makefile           
|-- parser             
|-- semant            
`-- cgen              
\end{verbatim}

\subsubsection{环境配置过程}
\textcolor{red}{%
% TODO: 简要描述你的环境配置过程,如果遇到了问题以及如何解决的。
% 例如:安装依赖包、设置环境变量、创建符号链接等步骤。
}

\section{Flex词法分析器原理}

本节要求详细阐述Flex的工作原理和词法分析的理论基础。\textbf{(本节占20分,是评分重点!)}

\subsection{Flex工作流程}

\textcolor{red}{%
% TODO: 详细说明Flex从.flex文件到可执行词法分析器的完整流程:
% 1. 读取.flex文件(包含正则表达式规则)
% 2. Flex工具如何解析这些规则
% 3. 生成C/C++源代码(lex.yy.c或cool-lex.cc)
% 4. 编译链接生成最终的lexer可执行文件
% 5. 词法分析器如何读取源文件并输出Token序列
% 
% 建议画一个流程图来说明整个过程。
}

\subsection{有限状态自动机(FSA)原理}

\textcolor{red}{%
% TODO: 详细阐述有限状态自动机的原理:
% 1. 什么是有限状态自动机(Finite State Automaton)
% 2. NFA(非确定有限自动机)与DFA(确定有限自动机)的区别
% 3. Flex如何将正则表达式转换为NFA
% 4. 如何从NFA转换为DFA(子集构造法)
% 5. DFA的最小化过程
% 
% 建议:
% - 用一个简单的例子(如识别整数)画出对应的NFA和DFA
% - 说明状态、转换、接受状态等概念
% - 解释为什么DFA比NFA效率更高
}

\subsection{模式匹配机制}

\textcolor{red}{%
% TODO: 说明Flex如何进行模式匹配:
% 1. 最长匹配原则(Longest Match)
% 2. 规则优先级(First Match,先定义的规则优先)
% 3. 如何处理二义性(例如关键字vs标识符)
% 4. yytext、yyleng等变量的作用
% 5. 输入缓冲区的管理
% 
% 举例说明:当输入"class"时,为什么匹配CLASS而不是标识符?
}

\subsection{状态与状态转换}

\textcolor{red}{%
% TODO: 详细解释Flex的状态机制:
% 1. 什么是Flex的状态(State)
% 2. INITIAL状态、独占状态(\%x)、包容状态(\%s)的区别
% 3. BEGIN()宏如何实现状态转换
% 4. 为什么需要状态(以字符串和注释处理为例)
% 5. 状态转换的具体过程(画状态转换图)
% 
% 建议:以你实现的STRING状态为例,画出状态转换图,说明:
% - 如何从INITIAL进入STRING
% - 在STRING状态中如何处理不同字符
% - 如何返回INITIAL状态
}


\section{实现细节}
本节简要说明词法规则的实现思路。完整代码见附录。

\subsection{关键字与标识符}

\textbf{基本要求:}
\begin{itemize}
    \item 关键字(如class、if、while等)是大小写不敏感的
    \item 布尔常量true和false的首字母必须小写
    \item TYPE\_ID以大写字母开头,OBJECT\_ID以小写字母开头
    \item 整数常量为数字序列
\end{itemize}

\textcolor{red}{%
% TODO: 说明你如何实现关键字的大小写不敏感、如何区分两种标识符。
% 特别说明:如何处理关键字与标识符的优先级问题。
}


\subsection{字符串处理}

\textbf{基本要求:}
\begin{itemize}
    \item 字符串以双引号开始和结束,不能跨行
    \item 支持转义字符:\texttt{\textbackslash n}, \texttt{\textbackslash t}, \texttt{\textbackslash b}, \texttt{\textbackslash f}, \texttt{\textbackslash"}, \texttt{\textbackslash\textbackslash}
    \item 最大长度1024字符,不能包含空字符
    \item 需要使用Flex状态机(\texttt{\%x STRING})处理
\end{itemize}

\textcolor{red}{%
% TODO: 详细说明你的字符串状态机设计:
% 1. 如何进入和退出STRING状态
% 2. 如何处理各种转义字符
% 3. 如何检测和报告错误(未闭合、过长、包含空字符、EOF等)
% 4. 如何使用字符串缓冲区
}

\subsection{操作符与注释}

\textbf{基本要求:}
\begin{itemize}
    \item 多字符操作符:\texttt{<-}、\texttt{<=}、\texttt{=>}
    \item 单行注释:\texttt{--}到行尾
    \item 多行注释:\texttt{(* *)},支持嵌套
    \item 忽略空白字符,换行时更新行号
\end{itemize}

\textcolor{red}{%
% TODO: 说明你如何处理:
% 1. 多字符操作符与单字符操作符的优先级
% 2. 嵌套注释的实现(使用计数器跟踪嵌套层数)
% 3. 行号的正确更新
}

\subsection{错误处理}

\textbf{需要检测的错误:}
\begin{itemize}
    \item 未闭合的字符串、字符串过长、字符串中的空字符
    \item EOF在字符串或注释中
    \item 未匹配的注释结束符\texttt{*)}
    \item 源代码中的非法字符
\end{itemize}

\textcolor{red}{%
% TODO: 说明你如何检测和报告各种错误,如何设置error\_msg,如何从错误中恢复。
}

\section{测试与验证}
为了验证词法分析器的正确性,我设计了多个测试用例,并使用了项目提供的测试工具。

\subsection{基础功能测试}

测试目标:验证关键字、标识符、常量、操作符的正确识别。

\textcolor{red}{%
% TODO: 编写测试文件test.cl,运行 ./lexer test.cl 或 make dotest,将输出贴在下方。
}

\textbf{测试用例}(test\_basic.cl):
\begin{lstlisting}[language=cool, caption={基础功能测试}]
class Main {
    x : Int <- 42;
    flag : Bool <- true;
    main() : Int { x };
};
\end{lstlisting}

\textbf{测试命令:}
\begin{verbatim}
$ ./lexer test_basic.cl
\end{verbatim}

\textbf{实际输出结果:}
\textcolor{red}{%
% TODO: 将实际的lexer输出贴在这里,应该类似:
}
\begin{verbatim}
#name "test_basic.cl"
#1 CLASS
#1 TYPEID Main
#1 '{'
#2 OBJECTID x
#2 ':'
#2 TYPEID Int
#2 ASSIGN
#2 INT_CONST 42
#2 ';'
#3 OBJECTID flag
#3 ':'
#3 TYPEID Bool
#3 ASSIGN
#3 BOOL_CONST true
#3 ';'
#4 OBJECTID main
...
\end{verbatim}

\subsection{字符串与注释测试}

测试目标:验证字符串转义字符、嵌套注释、各种错误检测。

\textcolor{red}{%
% TODO: 编写包含字符串和注释的测试文件,运行词法分析器,将输出贴在下方。
}

\textbf{测试用例}(test\_string.cl):
\begin{lstlisting}[language=cool, caption={字符串测试}]
(* 测试注释 (* 嵌套注释 *) *)
class Test {
    str1 : String <- "Hello\nWorld";  -- 转义字符
    str2 : String <- "Quote\"Test\"";
};
\end{lstlisting}

\textbf{测试命令与输出:}
\begin{verbatim}
$ ./lexer test_string.cl
\end{verbatim}

\textcolor{red}{%
% TODO: 贴上实际输出,验证:
% 1. 注释被正确忽略
% 2. 字符串正确识别,转义字符处理正确
% 3. STR_CONST token包含转义后的内容
}
\begin{verbatim}
#name "test_string.cl"
#2 CLASS
#2 TYPEID Test
#2 '{'
#3 OBJECTID str1
#3 ':'
#3 TYPEID String
#3 ASSIGN
#3 STR_CONST "Hello\nWorld"
...
\end{verbatim}

\subsection{错误处理测试}

测试目标:验证各种错误情况的检测和报告。

\textcolor{red}{%
% TODO: 测试各种错误情况,每个错误单独测试,展示错误输出。
}

\textbf{测试1:未闭合字符串}
\begin{verbatim}
测试代码:class Main { str : String <- "unclosed
输出:#1 ERROR "Unterminated string constant"
\end{verbatim}

\textbf{测试2:未匹配的注释结束符}
\begin{verbatim}
测试代码:class Main { x : Int; *) }
输出:#1 ERROR "Unmatched *)"
\end{verbatim}

\textbf{测试3:EOF在注释中}
\begin{verbatim}
测试代码:(* comment without closing
输出:#1 ERROR "EOF in comment"
\end{verbatim}

%TODO:  其他内容

\subsection{集成测试}

\textbf{本部分占15分,是评分重点!}

测试目标:验证词法分析器能与编译器其他阶段(语法分析、语义分析、代码生成)正确协作,最终生成可运行的MIPS汇编代码。

\textcolor{red}{%
% TODO: 编写多个完整的COOL程序,展示完整的编译和运行过程:
% 1. 至少包含2-3个不同复杂度的测试程序
% 2. 展示完整的编译命令和输出
% 3. 展示SPIM运行结果
% 4. 说明词法分析器在整个编译流程中的作用
% 5. 如果遇到问题,说明如何调试和解决
}

\textbf{测试程序}(hello.cl):
\begin{lstlisting}[language=cool, caption={集成测试程序}]
class Main inherits IO {
    main() : Object {
        out_string("Hello, COOL!\n")
    };
};
\end{lstlisting}

\textbf{编译过程:}
\begin{verbatim}
$ ./mycoolc hello.cl
\end{verbatim}

\textcolor{red}{%
% TODO: 如果编译成功,应该生成hello.s文件。贴出编译信息(如果有)。
% 如果有警告或错误,说明问题并解决方案。
}

\textbf{运行结果:}
\begin{verbatim}
$ spim hello.s
SPIM Version 8.0 of January 8, 2010
Copyright 1990-2010, James R. Larus.
All Rights Reserved.
Loaded: /usr/class/lib/trap.handler
Hello, COOL!
COOL program successfully executed
\end{verbatim}

\textbf{测试结论:}
\textcolor{red}{%
% TODO: 总结集成测试结果,说明词法分析器与编译器其他部分配合正常。
}

词法分析器成功识别了所有Token,编译器顺利完成了语法分析、语义分析和代码生成,生成的MIPS汇编代码在SPIM模拟器上正确执行,输出了预期结果。这证明词法分析器的实现是正确和完整的。

\section{遇到的问题与解决方案}
\textcolor{red}{%
% TODO: (可选,但建议填写)描述你在开发过程中遇到的主要挑战。
% 例如:
% 1. 对Flex状态机理解不深导致的问题。
% 2. 规则优先级问题(关键字 vs 标识符)。
% 3. C++与Flex的集成问题。
% 简单说明你是如何解决这些问题的。这能很好地体现你的学习和解决问题的能力。
}

\section{总结}
通过本次实验,我深入理解了词法分析的理论基础和Flex工具的工作原理。从有限状态自动机的理论出发,理解了Flex如何将正则表达式转换为高效的DFA,如何进行模式匹配和状态转换。在实践中,我成功实现了一个功能完整且健壮的COOL语言词法分析器,特别是掌握了状态管理机制在处理复杂词法结构时的应用。通过完整的集成测试,我验证了词法分析器能够与编译器其他组件正确协作,最终生成可执行的MIPS代码。这次实验让我对编译器前端有了全面而深刻的认识。

\textcolor{red}{%
% TODO: 根据你的真实感受修改总结。重点总结:
% 1. 对Flex原理的理解和收获
% 2. 对有限状态自动机理论的认识
% 3. 实现过程中的主要挑战和解决方法
% 4. 对整个编译流程的理解
}

\appendix
\section{附录: cool.flex 完整源码}
\label{sec:appendix_code}

\textcolor{red}{%
% TODO: 将你完成的cool.flex文件放在与此tex文件相同的目录下。
% 重新编译后,完整代码会自动包含在这里。
}

\IfFileExists{cool.flex}
  {\lstinputlisting[language=flex, basicstyle=\ttfamily\tiny]{cool.flex}}
  {
\textbf{说明:}将完成的 \texttt{cool.flex} 文件放在同一目录下,重新编译即可自动包含完整代码。

示例结构(3段式):定义段 -- 规则段 -- 用户代码段
}

\end{document}
